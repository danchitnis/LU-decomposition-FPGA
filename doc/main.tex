\documentclass[11pt,a4paper]{article}
\usepackage{Preamble}
\usepackage{subfiles}

\title{XILINX OPEN HARDWARE COMPETITION 2022: Acceleration of LU Decomposition on FPGAs}
\author{Yichen Zhang}
\date{\today}

\begin{document}
\maketitle
\begin{flushright}
    \textbf{Team number:} xohw22-006\\
    \textbf{Supervisor:} Dr Danial Chitnis
\end{flushright}

\section*{Abstract}
    A brief document to illustrate how to use \LaTeX.

\section{Introduction}
\subsection{Sparse Linear System}
Solving a linear system \(Ax=b\) is relatively a basic algorithm problem and is closely related to the applications of almost every field, including engineering, physics, chemistry, etc. It is at the heart of various algorithms and applications, including computational geometry, circuit simulation and data science. A variety of approaches have been developed to solve the linear system, which can mainly be classified into two types: direct methods and iterative methods.

The direct methods solve the sparse linear system mainly by computing the determinant, inversion or factorisation of a matrix. For example, Cramer's rule calculates the solution in terms of the determinants of the coefficient matrix and the right-hand-side vector. Gaussian elimination finds the solution by calculating the inverse of the system. LU decomposition solves the system by factorising the system into a lower triangular matrix \(L\) and an upper triangular matrix \(U\).

On the other hand, the iterative method is more modern. It attempts to solve the linear system based on an initial condition and successive approximates to the final solution, such as spectral sparsificaiton of graphs \cite{Spielman2008} and division-free inversion \cite{Koutis2012}. However, the behaviour and its stability of iterative methods is heavily based on the convergence\footnote{For an \(n\times n\) matrix \emph{A}, \emph{A} is convergent if: \vspace{-1ex}\[\forall i,j\in[1,n],i,j\in\mathbb{N}\Rightarrow \lim_{k\to\infty}\left(A^k\right)_{ij}=0\]} of the input matrices. On the contrary, the operation cost of behaviour methods is closely related to the cost of the matrix multiplication, making it preferred to the randomised and iterative methods in real applications as the directed methods are more robust and predictable.

LU decomposition is a direct method that can solve a large sparse linear systems multiple times, with various applications in circuit simulation, structure analysis, power networks, etc. A variety of parallel sparse linear solvers have adopted LU decomposition and have been running on massively parallel supercomputers. For example, Cray XE6 \cite{Kwack2016}, which is a type of distributed-memory machines, has implemented SuperLU\textunderscore DIST \cite{Li2003} solver for LU decomposition.

With the continuous development of IC industry, the size of FPGAs has grown to the extent that intense and heavy floating-point operations can be now accommodated. After a decade of research, it has been proved that the accelerating algorithms on FPGAs is a promising research avenue. Modern FPGAs have made it possible to fit a large computation kernel and establish parallel computation machines. In this paper, a sparse linear solver is build based on a CPU-FPGA architecture. While the pre-processing is made on CPU, the FPGA performs the numeric factorisation and solving of the matrices.

\subsection{Project Aim}


\subsection{Contribution}
\begin{itemize}
    \item The contribution of the team is to design a sparse linear solver based on a CPU-FPGA architecture.
    \item
\end{itemize}

Our responses to the Xilinx Open Hardware 2022 are given below:
\begin{itemize}
    \item \textbf{Technical Complexity:}
    \item \textbf{Implementation:}
    \item \textbf{Marketability/Innovation:}
    \item \textbf{Re-usability:}
\end{itemize}

\section{Foundation of LU Decomposition}
Suppose \(A\in\mathbb{R}^{n\times n}\). An LU decomposition of \(A\) refers to the factorisation of \(A\) into a lower triangular matrix \(L\) and an upper triangular matrix \(U\), with proper row and/or column permutations. Here, permutation is to change the order of the row or column of a matrix according to a row permutation matrix \emph{P} and a column permutation matrix \emph{Q}, and each row and column of the permutation matrix contains a single 1 with 0s everywhere else.

A sparse matrix refers to a matrix that has few nonzeros in it. Although the quantification of ``few'' is not defined, typically \(\mathcal{O}(n)\) elements are in a sparse matrix, where \(n\) is the order of the matrix. Sparse matrices are omnipresent in scientific computation when modelling systems with numbers of elements with restricted couplings.

\section{Implementation}
The factorisation phase of the LU decomposition is implemented on the CPU and the solving phase is implemented in the FPGA. In this work,

\subsection{HLS Configuration}
\begin{itemize}
    \item \textit{\# pragma HLS ARRAY\textunderscore PARTITION variable = Xwork type = block factor = 32 dim = 2}\\
    An array that is implemented in the BRAM may subject to the limit port access, which prevent the parallel access to the same array. A dual-port RAM can also allow two simultaneous access in one clock cycle.Therefore, arrays which are implemented as memory or memory ports can frequently create performance bottlenecks.

    With array partition, it is easier to implement the vectorisation optimisation, which typically requires parallel access to the same array.

    \item \textit{\# pragma HLS UNROLL factor = 32}\\
    Loop unrolling can create multiple separate operations instead of a single set of operations. The \emph{UNROLL} pragma changes loops by duplicating the loop body in the RTL design, allowing part of or entire loop iterations to run in parallel. By default, loops in C/C++ functions are kept rolled.  Synthesis builds the logic for one iteration of the loop when loops are rolled, and the RTL design executes this logic in order for each iteration of the loop. The loop induction variable specifies the number of iterations for which the loop should be run. Logic inside the loop body, such as break conditions or changes to a loop exit variable, can also affect the number of iterations. The \emph{UNROLL} pragma can be used to increase data access and throughput.
\end{itemize}

\section{Results}


\section{Conclusion and Future Work}

\renewcommand*{\bibname}{References}

\begin{sloppypar}
  \hbadness 10000\relax
  \printbibliography[heading=bibintoc,title={References}]
\end{sloppypar}

\end{document}